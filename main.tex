% !TeX program = xelatex
\RequirePackage{pdfmanagement-testphase}
\DeclareDocumentMetadata{uncompress,pdfstandard=A-1b}
% edition = debug -> blindreview -> electronic (default) -> print
% header = true/false 是否展示页眉,默认为 false
\documentclass[edition=electronic,header=false]{scnuthesis}

\usepackage{zhlipsum,lipsum} % 正式文章中请删去这两个宏包

\scnusetup{
  分类号 = {},
  学校代码 = {10574},
  UDC = {},
  密级 = {公开},
  学号 = {2020020000},
  学位类型 = {学术}, % 学术, 专业, 直博, 普通
  培养层次 = {硕士}, % 硕士, 博士
  题目 = {华南师范大学硕士\ / 博士学位论文\LaTeX3模板使用手册},
  title = {How to use the \LaTeX3 Document Class for SCNU Dissertations},
  姓名 = {张三},
  name = {Zhang San},
  手机号 = {13112345678},
  email = {123456789@qq.com},
  专业名称 = {基础数学},
  major = {Basic mathematics},
  研究方向 = {不定方程与超越数论},
  学院 = {数学科学学院},
  % 目前只支持单导师
  导师姓名 = {李四},
  tutor = {Li Si},
  导师职称 = {教授},
  导师手机 = {13800138000},
  date = {\today},
  bib文件名 = {ref.bib},
  % 所有图片、扫描件都要放在 ./fig/ 文件夹下
  签名图 = {signature.pdf},
%  答辩通过证明 = {答辩合格证明.pdf},
%  原创授权声明 = {statement.jpg},
}

% 粗体标签、宋体内容样式
\addtheorem[cndefinition]{
  definition = 定义,
  assume     = 假设,
  lemma      = 引理,
  question   = 问题,
}[chapter]
% 粗体标签、楷体内容样式
\addtheorem[cnplain]{
  exercise    = 练习,
  example     = 例,
  theorem     = 定理,
  proposition = 命题,
  corollary   = 推论,
  conjecture  = 猜想,
}[chapter]
% 楷体标签、宋体内容样式
\addtheorem[cnremark]{
  remark = 注,
}[chapter]


\begin{document}

\makecover

\electronicinput{data/passdefense}

\frontmatter

%% !TeX root = ../main.tex

\begin{cnabstract}
华南师范大学是一所有着悠久历史和深厚人文底蕴的高等学府,学校地处中国改革开放之都——广州,深得岭南开放务实之精神,有着素朴的传统和良好的学风,七十多年来,学校坚持师范教育的特色,始终致力于培养人格健全、有思想、有能力、有社会责任感的优秀人才。

在大学日益参与经济社会发展的新世纪,华南师范大学正以崭新的风貌、开阔的世界眼光,不断拓展大学的育人理念,创造良好的教学和包容的学术研究环境,营造丰富的校园文化,努力构建特色鲜明、开放式、综合性高水平教学研究型大学。
\end{cnabstract}

\cnkeywords{华南师范大学, 论文, \LaTeX 模板}

\begin{enabstract}
South China Normal University is an institution of higher education with a long history and a rich legacy. Situated in Guangzhou, the open metropolis in South China, South China Normal University is imbued with Lingnan's pioneering and pragmatic spirits. It has developed a tradition of elegant simplicity and fostered a strong learning environment. In the past seventy years or so, South China Normal University has preserved its characteristics of teacher education and has been devoted to cultivating talents with moral integrity, independent thinking, innovative ability and sense of social responsibility.

In the new century in which the institutions of higher education are getting increasingly involved in the social and economic development, South China Normal University has adopted an international view on education. Having broadened its horizon on the key issue of cultivating talents, it has made its greatest efforts to create a congenial and harmonious environment for both teaching and academic research and has fostered a rich variety of campus culture. It aims to build itself into a high-level comprehensive teaching and research-oriented university with open distinctive features.
\end{enabstract}

\enkeywords{SCNU; thesis; \LaTeX\ template}


\tableofcontents
% \listoftables
% \listoffigures

\mainmatter

%% !TeX root = ../main.tex

\chapter{绪论}
\section{本模板的意义}
\subsection{现有的学位论文模板的不足}
\zhlipsum\footnote{123132}\footnote{987987}
\subsubsection{\LaTeX 的特性}
\zhlipsum\footnote{456456}
\begin{equation}
p(y|\boldsymbol{x}) = \frac{p(\boldsymbol{x},y)}{p(\boldsymbol{x})}=
\frac{p(\boldsymbol{x}|y)p(y)}{p(\boldsymbol{x})} 
\end{equation}
%% !TeX root = ../main.tex

\chapter{公式定理}
\label{sec:equation}
贝叶斯公式如式~\eqref{equ:chap1:bayes},其中 $p(y|\boldsymbol{x})$ 为后验;
$p(\boldsymbol{x})$ 为先验;分母 $p(\boldsymbol{x})$ 为归一化因子\cite{刘建军2013GAP}。图~\ref{fig:1},表~\ref{tab:1}。
\begin{equation}
\label{equ:chap1:bayes}
p(y|\boldsymbol{x}) = \frac{p(\boldsymbol{x},y)}{p(\boldsymbol{x})}=
\frac{p(\boldsymbol{x}|y)p(y)}{p(\boldsymbol{x})} 
\end{equation}

\begin{figure}[!htb]
\centering
\includegraphics[width=.5\linewidth]{example-image-a}
\caption{巴拉巴拉}
\label{fig:1}
\end{figure}

\begin{table}[!ht]
\centering
\caption{巴拉巴拉}
\label{tab:1}
\begin{tabular}{cc}
\toprule
aa & bb \\
\midrule
cc & dd \\
ee & ff \\
\bottomrule
\end{tabular}
\end{table}

\printbibliography[heading=bibintoc]

\appendix

%% !TeX root = ../main.tex

\chapter{外文资料原文}
\section{first principles}
\subsection{Typography exists to honor content}
\lipsum
%% !TeX root = ../main.tex

\chapter{其他附录}

\backmatter

\unblindinput{data/resume}
\unblindinput{data/acknowledgement}

\statement

% \publication

\end{document}
