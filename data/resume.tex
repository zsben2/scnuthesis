%% !TeX root = ../main.tex

\chapter{作者攻读学位期间发表的学术论文}


\begin{refsection}[自己的著作.bib]
\nocite{*}
\printbibliography[
  heading = none,
  keyword = paper,
  resetnumbers = true
]
\end{refsection}




% \section*{发表的学术论文}

% \begin{enumerate}[label={[\arabic*]}]
% \item Yang Y, Ren T L, Zhang L T, et al. Miniature microphone with silicon-
%     based ferroelectric thin films. Integrated Ferroelectrics, 2003,
%     52:229-235. (SCI 收录, 检索号:758FZ.)
% \item 杨轶, 张宁欣, 任天令, 等. 硅基铁电微声学器件中薄膜残余应力的研究. 中国机
%     械工程, 2005, 16(14):1289-1291. (EI 收录, 检索号:0534931 2907.)
% \item 杨轶, 张宁欣, 任天令, 等. 集成铁电器件中的关键工艺研究. 仪器仪表学报,
%     2003, 24(S4):192-193. (EI 源刊.)
% \end{enumerate}

% \begin{refsection}[自己的著作.bib]
% \nocite{*}
% \printbibliography[
%   heading = subbibliography,
%   title = 发表的学术论文,
%   keyword = paper,
%   resetnumbers = true
% ]
% \end{refsection}


% \section*{研究成果}

% \begin{enumerate}[label={[\arabic*]}]
% \item 任天令, 杨轶, 朱一平, 等. 硅基铁电微声学传感器畴极化区域控制和电极连接的
%     方法: 中国, CN1602118A. (中国专利公开号.)
% \item Ren T L, Yang Y, Zhu Y P, et al. Piezoelectric micro acoustic sensor
%     based on ferroelectric materials: USA, No.11/215, 102. (美国发明专利申请号.)
% \end{enumerate}

% \begin{refsection}[自己的著作.bib]
% \nocite{*}
% \printbibliography[
%   heading = subbibliography,
%   title = 研究成果,
%   keyword = software,
%   resetnumbers = true
% ]
% \end{refsection}

